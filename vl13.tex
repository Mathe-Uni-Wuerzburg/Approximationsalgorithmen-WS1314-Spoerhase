\begin{section}{Steinerwald-Problem}
 \underline{Eingabe:} Graph $G=(V,E)$, Kosten $c: E \mapsto \mathbb{N}$, $k$ Knotenpaare $s_i,t_i \in V$.\\
 \underline{Ziel:} Finde kostengünstigste Menge $F\subseteq E$, so dass in $(V,F)$ jedes der Paare $s_i,t_i$ verbunden ist.\\
 \underline{ILP:} 
 \begin{align*} \min &\sum_{e \in E} c_e\cdot x_e\\
 \text{s.t. } &\sum_{e\in \delta(s)} x_e \geq 1 &S\in S_i \text{ für ein }i\\
 &x_e \in \{0,1\} &\forall e\in E\\
 \end{align*}
 \underline{LP-Relaxierung:} $x_e \in \{0,1\} \mapsto x_e \geq 0$\\
 $S_i := \{S\subseteq V: |S\cap \{s_i,t_i\}| = 1\}$\\
 \underline{Duales LP:} 
 \begin{align*}
  \max &\sum_{S: \exists i: S \subseteq S_i} y_s\\
  \text{s.t.} &\sum_{S: e\in \delta(S)} y_s\leq c_e &\forall e\in E\\
  &y_s \geq 0 &\exists i: S\in S_i\\                          
 \end{align*}
 \underline{Algorithmus:} 
 
 \begin{lstlisting}[mathescape]
y $\leftarrow$ 0
F $\leftarrow \emptyset$
e $\leftarrow$ 0
while nicht alle Paare $s_i,t_i$ sind verbunden:
  e $\leftarrow$ e+1
  $\mathcal{C}:= \{$Zhgs.komponente C von (V,F)$| \exists i: |C\cap \{s_i,t_i\}| = 1\}$
  erhoehe $y_e$ fuer alle $c\in \mathcal{C}$ gleichfoermig, 
    bis fuer ein $e_l \in \delta(c'),c'\in \mathcal{C}$ gilt: 
    $c_{e_l} = \sum_{S: e_l\in \delta(s)} y_s$
  F$\leftarrow$ F + $e_l$
F' $\leftarrow$ F
for k $\leftarrow$ l to 1:
  if F' - $e_k$ zulässig: F' $\leftarrow$ F' - $e_k$
 \end{lstlisting}
 \underline{Beobachtung:} Die konstruierte Primal- und Duallösung ist zulässig.
 
 \begin{lemma}[13.1]
 \label{131}
  Für jedes $\mathcal{C}$ in jeder Iteration gilt:
  \[\sum_{c\in\mathcal{C}} |\delta(c) \cap F'| \leq 2\cdot |\mathcal{C}|\text{.}\]
 \end{lemma}
 Beweis später.
 \begin{satz}[13.2]
  Obiger Algorithmus ist eine 2-Approximation.
 \end{satz}
 \begin{proof}
  $\sum_{e\in F'} c_e = \sum_{e\in F'} \sum_{S: e\in\delta(S)} y_S = \sum_S |\delta(S)\cap F'| \cdot y_S$\\
  Zeigen per Induktion über di Anzahl der Iterationen, dass 
  \[\sum_S |F'\cap \delta(S)| \cdot y_S \leq 2\cdot \sum_S y_S \text{.   (1)}\]
  Daraus folgt die Behauptung über die schwache Duallösung. (1) gilt zu Beginn wegen $y=0$.\\
  Angenommen, dass (1) zu Beginn einer Iteration gilt. In diesen Iterationen werden alle $y_c, c\in \mathcal{C}$ um den gleichen Betrag, sagen wir $\varepsilon \geq 0$, erhöht. Die Erhöhung der beiden Seiten von (1) ist somit $\varepsilon \sum_{c\in\mathcal{C}} |\delta(c) \cap F'|$. Die Erhöhung der rechten Seite ist $2\cdot \varepsilon \cdot |\mathcal{C}|$. Nach Lemma \ref{131} gilt die Ungleichung also auch nach der Iteration.
 \end{proof}
 
 \begin{proposition}[Beobachtung]
  Zu jedem Zeitpunkt bildet die Kantenmenge $F$ einen Wald.
 \end{proposition}
 \begin{proof}
  Per Induktion über die Anzahl der Iterationen:\\
  Beginn: $F = \emptyset$. Jede Kante, die wir neu hinzufügen verbindet zwei Zusammenhangskomponenten von $F$.
 \end{proof}
 
 \begin{proof}[Proof zu Lemma \ref{131}]
  Betrachte Iteration $i$:\\
  Sei $F_i = \{e_1,\cdots,e_{i-1}\}$ die Menge der Kanten, die zu Beginn der Iteration schon im Wald sind. Betrachte die Menge noch dazukommenden Kanten $H=F'-F_i$. Beobachte, dass $F_i \cup H \supseteq F'$ eine zulässige Lösung ist. Wir behaupten, dass wenn wir eine Kante $e\in H$ aus $F_i \cap H$ entfernen, so ist die Lösung nicht mehr zulässig. Das liegt an den Löschprozeduren am Ende des Algorithmus. Wenn $e_{i-1}$ für die Löschung betrachtet wurde, gilt $F' = F' \cup H$. Daher sind alle bereits betrachteten Kanten notwendig. Dies sind genau die Kanten in $H$.\\
  Wir bilden aus $(V,F_i \cup H)$ einen neuen Graphen indem wir die Zusammenhangskomponenten von $(V,F_i)$ zu je einem Knoten kontrahieren. Die Kantenmenge des Graphen ist $H$. Seine Kontenmenge sei $V'$. $(V',H)$ ist ein Wald. Wir färben alle Knoten von $V'$ ($\Leftrightarrow$ Zusammenhangskomponenten von $(V,F_i)$). Alle Knoten, die zu Zusammenhangskomponenten von $\mathcal{C}$ korrespondieren, werden rot gefärbt. Alle restlichen Knoten werden blau gefärbt. Sei $R$ die Menge aller roten Knoten in $V'$ und $B$ die Menge der blauen Knoten $v$ mit $deg(v) > 0$.\\
  Die behauptete Ungleichung lässt sich umschreiben:\\
  Die rechte Seite ist $2|R|$. Die linke Seite ist $\sum_{v\in R} deg(v)$. Wir behaupten, dass keine Knoten in $B$ genau Grad $1$ hat. Dann folgt:
  \[\sum_{v\in R} deg(v) = \sum_{v \in R \cup B} deg(v) -\sum_{v \in B} deg(v)\leq 2\cdot(|R|+|B|) - 2 \cdot |B| = 2 \cdot |R|\]
  Angenommmen, ein blauer Knoten $v$ habe Grad $1$ und sei $\mathcal{C}$ die korrespondiere Zusammenhangskomponente und $e$ die inzidente Kante in $H$. Da $e$ notwendig ist, muss sie auf einem $(s_j,t_j)$-Pfad in $(V,F_i\cup H)$ liegen.\\
  Das heißt, $|C \cap \{s_i,t_i\}| = 1$ und somit $c \in \mathcal{C}$. $\lightning$
 \end{proof}
\end{section}